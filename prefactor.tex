%% LyX 2.0.4 created this file.  For more info, see http://www.lyx.org/.
%% Do not edit unless you really know what you are doing.
\documentclass[english]{article}
\usepackage[T1]{fontenc}
\usepackage[utf8]{luainputenc}
\usepackage{esint}
\usepackage{babel}
\begin{document}
One limitation of the step function model is that a prior regulazier
$Prior(\alpha)=exp(-\xi\sum(\alpha_{i}-\alpha_{i-1})^{2})$ is needed
to infer the P($\epsilon$) and g correctly and the constant prefactor
$\xi$ needs to be tuned.In our test the true P($\epsilon$) is set
to be a smooth function ,which is a natural assumption for P($\epsilon$).The
effect of this prior regulazier is to keep the inferred P($\epsilon$)
smooth. if the constant $\xi$ is not big enough the inferred P($\epsilon$)
will have an oscilation noise and the accuracy of g inference will
also decreases. Thus an empirical formula for $\xi$ is helpful to
tune this prefactor by simulations. from several tests we find that
$\xi$ is proportional to the mean data density and can be approximated
by this empirical formula.
\begin{equation}
\xi=<\rho>/256
\end{equation}


Here $<\rho>=\int Nf^{2}(x)dx$ is the mean data density. With the
help of this empirical formula,$\xi$ can be tuned by doing some simulation
which has similar data distribution with the real data.
\end{document}
