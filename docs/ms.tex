%%%%%%%%%%%%%%%%%%%%%%%%%%%%%%%%%%%%%%%%%%%%%%%%%%%%%%%%%%%%%%%%%%%%%%%%
%
% Hierarchical Inference in Weak Lensing: Inferring Pr(e)
%
%%%%%%%%%%%%%%%%%%%%%%%%%%%%%%%%%%%%%%%%%%%%%%%%%%%%%%%%%%%%%%%%%%%%%%%%

\documentclass[useAMS,usenatbib]{mn2e}
%% letterpaper
%% a4paper

% \voffset=-0.8in

% Packages:
% \input psfig.sty
\usepackage{xspace}
\usepackage{graphicx}

% Macros:
% JOURNALS
\newcommand {\apj} {ApJ}
\newcommand {\apjl} {ApJL}
\newcommand {\apjs} {ApJS}
\newcommand {\mnras} {MNRAS}
\newcommand {\apss} {Ap \& SS}
\newcommand {\aap} {A\&A}
\newcommand {\aj} {AJ}
\newcommand {\prd} {Phys. Rev. D}
\newcommand {\nat} {Nature}
\newcommand {\araa} {ARA\&A}
\newcommand {\jgr} {J. Geophys. Res.}
\newcommand {\pasp} {PASP}

% MISC
\newcommand {\etal} {et~al.~}
\def \spose#1{\hbox  to 0pt{#1\hss}}  
\newcommand {\lta} {\mathrel{\spose{\lower 3pt\hbox{$\sim$}}\raise  2.0pt\hbox{$<$}}}
\newcommand {\gta} {\mathrel{\spose{\lower  3pt\hbox{$\sim$}}\raise 2.0pt\hbox{$>$}}}
\def \ion#1#2{#1{\footnotesize{#2}}\relax}
\newcommand {\ha}  {\ifmmode H\alpha \else H$\alpha $ \fi} 
\newcommand {\hi} {\ion{H}{I} } 

\def\Sref#1{Section~\ref{#1}\xspace}
\def\Fref#1{Figure~\ref{#1}\xspace}
\def\Tref#1{Table~\ref{#1}\xspace}
\def\Eref#1{Equation~\ref{#1}\xspace}
\def\Eqref#1{Eq.~(\ref{#1})\xspace}

% UNITS
\newcommand {\kms} {\ifmmode  \,\rm km\,s^{-1} \else $\,\rm km\,s^{-1}  $ \fi }
\newcommand {\kpc} {\ifmmode  {\rm kpc}  \else ${\rm  kpc}$ \fi  }  
\newcommand {\pc} {\ifmmode  {\rm pc}  \else ${\rm pc}$ \fi  }  
\newcommand {\Msun} {\ifmmode {\rm M_{\odot}} \else ${\rm M_{\odot}}$ \fi} 
\newcommand {\Zsun} {\ifmmode {\rm Z_{\odot}} \else ${\rm Z_{\odot}}$ \fi} 
\newcommand {\yr} {\ifmmode yr^{-1} \else $yr^{-1}$ \fi} 
\newcommand {\hMsun} {\ifmmode h^{-1}\,\rm M_{\odot} \else $h^{-1}\,\rm M_{\odot}$ \fi}

% COSMOLOGY
%\newcommand {\LCDM} {\ifmmode \Lambda{\rm CDM} \else $\Lambda{\rm CDM}$ \fi}

% LENSING
\def\zd{z_{\rm d}}
\def\zs{z_{\rm s}}
\def\zspdf{z_{\rm s,pdf}\;}
\def\Dd{D_{\rm d}}
\def\Ds{D_{\rm s}}
\def\Dds{D_{\rm ds}}
\def\Sigmacrit{\Sigma_{\rm crit}}

% SOFTWARE/HARDWARE
\def\SExtractor{{\sc SExtractor}\xspace}
\def\hst{{\it HST}\xspace}
\def\acs{\hst/ACS\xspace}
\def\galfit{{\sc galfit}\xspace}
\def\idl{{\sc idl}\xspace}
\def\python{{\sc python}\xspace}

% PROBABILITY THEORY
\def\pr{{\rm Pr}}
\def\data{{\mathbf{d}}}
\def\datap{{\mathbf{d}^{\rm p}}}
\def\datai{d_i}
\def\datapi{d^{\rm p}_i}
\def\pars{\boldsymbol{\theta}}

% COMMENTING
\usepackage[usenames]{color}
\newcommand{\phil}[1]{\textcolor{blue}{\bf #1}}
\newcommand{\hogg}[1]{\textcolor{green}{\bf #1}}
\newcommand{\flag}[2]{\textcolor{red}{\it\bf #1: #2}}

\def\oxford{Department of Physics, University of Oxford, Keble Road, Oxford, OX1 3RH, UK}
\def\nyu{Center for Cosmology and Particle Physics, New York University, NY, USA}



%%%%%%%%%%%%%%%%%%%%%%%%%%%%%%%%%%%%%%%%%%%%%%%%%%%%%%%%%%%%%%%%%%%%%%%%

\title[Hierarchical Inference in Weak Lensing]
{Hierarchical Inference of the Intrinsic Ellipticity
Distribution of Galaxies during Weak Lensing Analysis}
    
\author[Tang et al]{%
  Yike~Tang$^{1}$,
  David W. Hogg,$^{1}$
  Philip~J.~Marshall$^{2}$
  \medskip\\
  $^1$\nyu\\
  $^2$\oxford
}

%%%%%%%%%%%%%%%%%%%%%%%%%%%%%%%%%%%%%%%%%%%%%%%%%%%%%%%%%%%%%%%%%%%%%%%%

\begin{document}
             
\date{to be submitted to MNRAS}
             
\pagerange{\pageref{firstpage}--\pageref{lastpage}}\pubyear{2010}

\maketitle           

\label{firstpage}

%%%%%%%%%%%%%%%%%%%%%%%%%%%%%%%%%%%%%%%%%%%%%%%%%%%%%%%%%%%%%%%%%%%%%%%%
% in this abstract, below, check the use of -, --, and ---.  Understand?

\begin{abstract}
Weak-lensing projects---to measure galaxy--galaxy lensing or the shear--shear
correlation function---often make use of estimators that involve simple or weighted
means of measured galaxy ellipticities.  Here we propose moving to a probabilistic inference
framework for weak-lensing projects, in which the both the distribution of unlensed
(intrinsic) galaxy ellipticities and the distribution of ellipticity measurement noise contributions
are both treated as priors.  We perform hierarchical inference to \emph{infer} the
unlensed ellipticity distribution simultaneously with the shear map.  The best
shear-field estimators (point estimates), in this context, become
maximum-\emph{marginalized}-likelihood estimates, in which uncertainty about the
inferred unlensed distribution has been marginalized out.  We show that our proposed
estimators perform better than simple mean-ellipticity estimators both in terms of
bias and variance.  We also show that as the number of observed galaxies becomes
large, the performance of our proposed estimators approaches the performance of a
magical maximum-likelihood estimator that could be constructed if the true unlensed
ellipticity distribution were known \textit{a priori}.  Importantly, this work only
considers the \emph{catalog-level} problem of how to combine galaxy ellipticity
measurements; it does not consider the problem of making those measurements in the
first place.  In the end we argue that weak lensing projects ought to move beyond
point estimates and instead propagate full likelihood-function information about
the shear field forward into scientific analyses.  Only this could properly
propagate uncertainty without introducing additional catalog-generated variance.
\end{abstract}

% Full list of options at http://www.journals.uchicago.edu/ApJ/instruct.key.html

\begin{keywords}
  gravitational lensing
\end{keywords}

\setcounter{footnote}{1}

%%%%%%%%%%%%%%%%%%%%%%%%%%%%%%%%%%%%%%%%%%%%%%%%%%%%%%%%%%%%%%%%%%%%%%%%
%%  SECTION 1: INTRODUCTION
%%%%%%%%%%%%%%%%%%%%%%%%%%%%%%%%%%%%%%%%%%%%%%%%%%%%%%%%%%%%%%%%%%%%%%%%

\section{Introduction}
\label{sec:intro}

\flag{Yike}{Plan the sections of your paper, and write very brief notes on
what you want to say in each of them. You will find it helpful to divide
them into sub-sections, and even paragraphs. You may also want to put in
placeholder figures and tables; if you do, make sure to write the captions of
them. The itemized conclusions should be partial answers to a corresponding
list of questions in the introduction. You can write these now, as incomplete
statements: when you know what your findings are, you can complete them.}

% WL as cosmological and astrophysical probe. 

% Information is scarce. 
% Expensive, high precision future. Motivation.

% This paper: toy problem, exploring one aspect of analysis.

% History of shear estimation, and treatment of Pr(e).


This paper is organized as follows...


%%%%%%%%%%%%%%%%%%%%%%%%%%%%%%%%%%%%%%%%%%%%%%%%%%%%%%%%%%%%%%%%%%%%%%%%
%%  SECTION N: XXX
%%%%%%%%%%%%%%%%%%%%%%%%%%%%%%%%%%%%%%%%%%%%%%%%%%%%%%%%%%%%%%%%%%%%%%%%

\section{method}
\label{sec:XXX}


\subsection{Likelihood Functions}
If we know the ellipticity distribution of unlensed galaxies based
on some parasmeters $\alpha$,the likelihood of reduced shear $g$
and parameter $\alpha$ is like:

\begin{equation}
L(\vec{g},\alpha|\epsilon_{\ell})=p(\epsilon_{\ell}|\vec{g},\alpha)=P(\epsilon_{0}(\vec{g})|\alpha)*|\frac{\partial\epsilon_{0}}{\partial\epsilon_{\ell}}|
\end{equation}


Note that both $\epsilon_{0}$ and $\epsilon_{\ell}$are complex numbers.So
$|\frac{\partial\epsilon_{0}}{\partial\epsilon_{\ell}}|$ is the Jacobian
determinant term.Here the prior distribtuion we use for $\vec{g}$
is a 2Dflat distribution in the region $|g|<1$.Therefore $prior(\vec{g)}=\begin{cases}
1 & |\vec{g}|<1\\
0 & |\vec{g}|>1
\end{cases}$

The prior of $\alpha$ some how depends on the model we choose for
intrinsic ellipticity distribution .When we use the step function
model for P($\epsilon_{0}$),we find that we need to use a prior on
$\alpha$ .
\begin{equation}
Prior(\vec{\alpha}|\xi)=exp(-\xi\sum(\alpha_{i}-\alpha_{i-1})^{2})
\end{equation}


Given the likehood functions and priors we can also calculated the
marginalized likelihood function.
\begin{equation}
p(\epsilon_{\ell}|\vec{g})=\int p(\epsilon_{\ell}|\vec{g},\alpha)p(\alpha|\vec{g})d\alpha
\end{equation}


Since $p(\alpha)$ does not explicitly depends on $\vec{g}$ .$p(\alpha|\vec{g})$is
just $Prior(\vec{\alpha}|\xi)$.Therefore,we have marginalized likehood
\begin{equation}
p(\epsilon_{\ell}|\vec{g})=\int p(\epsilon_{\ell}|\vec{g},\alpha)Prior(\vec{\alpha}|\xi)d\alpha
\end{equation}


\subsection{model for P($\epsilon_{0}$)}
Two flexible models are tested for P($\epsilon_{0}$)
One is Beta function model and one is step function model.
While using step function model we use a prior regularzation on $\alpha$ 
\begin{equation}
Prior(\vec{\alpha}|\xi)=exp(-\xi\sum(\alpha_{i}-\alpha_{i-1})^{2})
\end{equation}


\subsection{MCMC Sampler}

While making the posterior samples we choose to use the classical Gibbs sampler.





%%%%%%%%%%%%%%%%%%%%%%%%%%%%%%%%%%%%%%%%%%%%%%%%%%%%%%%%%%%%%%%%%%%%%%%%
%%  SECTION N: XXX
%%%%%%%%%%%%%%%%%%%%%%%%%%%%%%%%%%%%%%%%%%%%%%%%%%%%%%%%%%%%%%%%%%%%%%%%

\section{Result}
\label{sec:XXX}

\subsection{Posterior samples}
We show the posterior samples of both g and $\alpha$ ,for both beta function and step function model.
\subsection{Bias Test}
We test the systematical bias of our estimator.
We plot $g_{1}-g_{1}^{true}$ VS $g_{1}^{true}$ ,and put on an error bar which equals standard deviation of 
marginalized likehood functon $p(Data|g_{1})$ ,and use a linear fit to determin the m,c value and their standard 
deviations.

\subsection{Accuracy Test}
We compare the mean square error of our estimator with average and the estimator that could be constructed if the true unlensed
ellipticity distribution were known.
We show how the mean square error varies with number of galaxies ,number of patches and so on.

\subsection{Accuracy Test}
We show some example how our method works when there is an Gaussian noise added on the observed $\epsilon$

\subsection{prefactor $\xi$}
Discuss about how $\xi$ is tuned ,the danger if $\xi$ is set to be too large or too small. 





%%%%%%%%%%%%%%%%%%%%%%%%%%%%%%%%%%%%%%%%%%%%%%%%%%%%%%%%%%%%%%%%%%%%%%%%
%%  SECTION N: XXX
%%%%%%%%%%%%%%%%%%%%%%%%%%%%%%%%%%%%%%%%%%%%%%%%%%%%%%%%%%%%%%%%%%%%%%%%

\section{Discussion}
\label{sec:XXX}


%%%%%%%%%%%%%%%%%%%%%%%%%%%%%%%%%%%%%%%%%%%%%%%%%%%%%%%%%%%%%%%%%%%%%%%%
%%  SECTION X: DISCUSSION
%%%%%%%%%%%%%%%%%%%%%%%%%%%%%%%%%%%%%%%%%%%%%%%%%%%%%%%%%%%%%%%%%%%%%%%%

\section{Discussion}
\label{sec:discuss}


%%%%%%%%%%%%%%%%%%%%%%%%%%%%%%%%%%%%%%%%%%%%%%%%%%%%%%%%%%%%%%%%%%%%%%%%
%%  SECTION X: CONCLUSIONS
%%%%%%%%%%%%%%%%%%%%%%%%%%%%%%%%%%%%%%%%%%%%%%%%%%%%%%%%%%%%%%%%%%%%%%%%

\section{Conclusions}
\label{sec:conclusions}

In summary, ...

\begin{itemize}

\item We ...

\item The ...

\item The ...

\end{itemize}



%%%%%%%%%%%%%%%%%%%%%%%%%%%%%%%%%%%%%%%%%%%%%%%%%%%%%%%%%%%%%%%%%%%%%%%%
%%  ACKNOWLEDGMENTS
%%%%%%%%%%%%%%%%%%%%%%%%%%%%%%%%%%%%%%%%%%%%%%%%%%%%%%%%%%%%%%%%%%%%%%%%

\section*{Acknowledgments}
 
We thank XXX for useful discussions and suggestions.

YT acknowledges ....
%
DWH acknowledges support from ...
% 
PJM was given support by the Royal 
Society in the form of a research fellowship.
%
% Code used? Links to repositories?


%%%%%%%%%%%%%%%%%%%%%%%%%%%%%%%%%%%%%%%%%%%%%%%%%%%%%%%%%%%%%%%%%%%%%%%%
%%  APPENDICES
% %%%%%%%%%%%%%%%%%%%%%%%%%%%%%%%%%%%%%%%%%%%%%%%%%%%%%%%%%%%%%%%%%%%%%%
% 
% \appendix
% 
% \section{}
% \label{sec:appendix}
% 
% 
%%%%%%%%%%%%%%%%%%%%%%%%%%%%%%%%%%%%%%%%%%%%%%%%%%%%%%%%%%%%%%%%%%%%%%%%
%%  REFERENCES
%%%%%%%%%%%%%%%%%%%%%%%%%%%%%%%%%%%%%%%%%%%%%%%%%%%%%%%%%%%%%%%%%%%%%%%%

% MNRAS does not use bibtex, input .bbl file instead. 
% Generate this in the makefile using bubble script in scriptutils:

%   bubble -f PS1QLS-survey.tex references.bib 

\bibliographystyle{apj}
\bibliography{references}
%\input{PS1QLS-survey.bbl}

%%%%%%%%%%%%%%%%%%%%%%%%%%%%%%%%%%%%%%%%%%%%%%%%%%%%%%%%%%%%%%%%%%%%%%%%

\label{lastpage}
\bsp

\end{document}

%%%%%%%%%%%%%%%%%%%%%%%%%%%%%%%%%%%%%%%%%%%%%%%%%%%%%%%%%%%%%%%%%%%%%%%%
