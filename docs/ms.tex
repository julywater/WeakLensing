%%%%%%%%%%%%%%%%%%%%%%%%%%%%%%%%%%%%%%%%%%%%%%%%%%%%%%%%%%%%%%%%%%%%%%%%
%
% hierarchical Inference in Weak Lensing: Inferring Pr(e)
%
%%%%%%%%%%%%%%%%%%%%%%%%%%%%%%%%%%%%%%%%%%%%%%%%%%%%%%%%%%%%%%%%%%%%%%%%

\documentclass[useAMS,usenatbib]{mn2e}
%% letterpaper
%% a4paper

% \voffset=-0.8in

% Packages:
% \input psfig.sty
\usepackage{xspace}
\usepackage{graphicx}

% Macros:
% JOURNALS
\newcommand {\apj} {ApJ}
\newcommand {\apjl} {ApJL}
\newcommand {\apjs} {ApJS}
\newcommand {\mnras} {MNRAS}
\newcommand {\apss} {Ap \& SS}
\newcommand {\aap} {A\&A}
\newcommand {\aj} {AJ}
\newcommand {\prd} {Phys. Rev. D}
\newcommand {\nat} {Nature}
\newcommand {\araa} {ARA\&A}
\newcommand {\jgr} {J. Geophys. Res.}
\newcommand {\pasp} {PASP}

% MISC
\newcommand {\etal} {et~al.~}
\def \spose#1{\hbox  to 0pt{#1\hss}}  
\newcommand {\lta} {\mathrel{\spose{\lower 3pt\hbox{$\sim$}}\raise  2.0pt\hbox{$<$}}}
\newcommand {\gta} {\mathrel{\spose{\lower  3pt\hbox{$\sim$}}\raise 2.0pt\hbox{$>$}}}
\def \ion#1#2{#1{\footnotesize{#2}}\relax}
\newcommand {\ha}  {\ifmmode H\alpha \else H$\alpha $ \fi} 
\newcommand {\hi} {\ion{H}{I} } 

\def\Sref#1{Section~\ref{#1}\xspace}
\def\Fref#1{Figure~\ref{#1}\xspace}
\def\Tref#1{Table~\ref{#1}\xspace}
\def\Eref#1{Equation~\ref{#1}\xspace}
\def\Eqref#1{Eq.~(\ref{#1})\xspace}

% UNITS
\newcommand {\kms} {\ifmmode  \,\rm km\,s^{-1} \else $\,\rm km\,s^{-1}  $ \fi }
\newcommand {\kpc} {\ifmmode  {\rm kpc}  \else ${\rm  kpc}$ \fi  }  
\newcommand {\pc} {\ifmmode  {\rm pc}  \else ${\rm pc}$ \fi  }  
\newcommand {\Msun} {\ifmmode {\rm M_{\odot}} \else ${\rm M_{\odot}}$ \fi} 
\newcommand {\Zsun} {\ifmmode {\rm Z_{\odot}} \else ${\rm Z_{\odot}}$ \fi} 
\newcommand {\yr} {\ifmmode yr^{-1} \else $yr^{-1}$ \fi} 
\newcommand {\hMsun} {\ifmmode h^{-1}\,\rm M_{\odot} \else $h^{-1}\,\rm M_{\odot}$ \fi}

% COSMOLOGY
%\newcommand {\LCDM} {\ifmmode \Lambda{\rm CDM} \else $\Lambda{\rm CDM}$ \fi}

% LENSING
\def\zd{z_{\rm d}}
\def\zs{z_{\rm s}}
\def\zspdf{z_{\rm s,pdf}\;}
\def\Dd{D_{\rm d}}
\def\Ds{D_{\rm s}}
\def\Dds{D_{\rm ds}}
\def\Sigmacrit{\Sigma_{\rm crit}}

% SOFTWARE/HARDWARE
\def\SExtractor{{\sc SExtractor}\xspace}
\def\hst{{\it HST}\xspace}
\def\acs{\hst/ACS\xspace}
\def\galfit{{\sc galfit}\xspace}
\def\idl{{\sc idl}\xspace}
\def\python{{\sc python}\xspace}

% PROBABILITY THEORY
\def\pr{{\rm Pr}}
\def\data{{\mathbf{d}}}
\def\datap{{\mathbf{d}^{\rm p}}}
\def\datai{d_i}
\def\datapi{d^{\rm p}_i}
\def\pars{\boldsymbol{\theta}}

% COMMENTING
\usepackage[usenames]{color}
\newcommand{\phil}[1]{\textcolor{blue}{\bf #1}}
\newcommand{\hogg}[1]{\textcolor{green}{\bf #1}}
\newcommand{\flag}[2]{\textcolor{red}{\it\bf #1: #2}}

\def\oxford{Department of Physics, University of Oxford, Keble Road, Oxford, OX1 3RH, UK}
\def\nyu{Center for Cosmology and Particle Physics, New York University, NY, USA}



%%%%%%%%%%%%%%%%%%%%%%%%%%%%%%%%%%%%%%%%%%%%%%%%%%%%%%%%%%%%%%%%%%%%%%%%

\title[Hierarchical Inference in Weak Lensing]
{Hierarchical Inference of the Intrinsic Ellipticity
Distribution of Galaxies during Weak Lensing Analysis}
    
\author[Tang et al]{%
  Yike~Tang$^{1}$,
  David W. Hogg,$^{1}$
  Philip~J.~Marshall$^{2}$
  \medskip\\
  $^1$\nyu\\
  $^2$\oxford
}

%%%%%%%%%%%%%%%%%%%%%%%%%%%%%%%%%%%%%%%%%%%%%%%%%%%%%%%%%%%%%%%%%%%%%%%%

\begin{document}
             
\date{to be submitted to MNRAS}
             
\pagerange{\pageref{firstpage}--\pageref{lastpage}}\pubyear{2010}

\maketitle           

\label{firstpage}

%%%%%%%%%%%%%%%%%%%%%%%%%%%%%%%%%%%%%%%%%%%%%%%%%%%%%%%%%%%%%%%%%%%%%%%%
% in this abstract, below, check the use of -, --, and ---.  Understand?

\begin{abstract}
Weak-lensing projects---to measure galaxy--galaxy lensing or the shear--shear
correlation function---often make use of estimators that involve simple or weighted
means of measured galaxy ellipticities.  Here we propose moving to a probabilistic inference
framework for weak-lensing projects, in which the both the distribution of unlensed
(intrinsic) galaxy ellipticities and the distribution of ellipticity measurement noise contributions
are both treated as priors.  We perform hierarchical inference to \emph{infer} the
unlensed ellipticity distribution simultaneously with the shear map.  The best
shear-field estimators (point estimates), in this context, become
maximum-\emph{marginalized}-likelihood estimates, in which uncertainty about the
inferred unlensed distribution has been marginalized out.  We show that our proposed
estimators perform better than simple mean-ellipticity estimators both in terms of
bias and variance.  We also show that as the number of observed galaxies becomes
large, the performance of our proposed estimators approaches the performance of a
magical maximum-likelihood estimator that could be constructed if the true unlensed
ellipticity distribution were known \textit{a priori}.  Importantly, this work only
considers the \emph{catalog-level} problem of how to combine galaxy ellipticity
measurements; it does not consider the problem of making those measurements in the
first place.  In the end we argue that weak lensing projects ought to move beyond
point estimates and instead propagate full likelihood-function information about
the shear field forward into scientific analyses.  Only this could properly
propagate uncertainty without introducing additional catalog-generated variance.
\end{abstract}

% Full list of options at http://www.journals.uchicago.edu/ApJ/instruct.key.html

\begin{keywords}
  gravitational lensing
\end{keywords}

\setcounter{footnote}{1}

%%%%%%%%%%%%%%%%%%%%%%%%%%%%%%%%%%%%%%%%%%%%%%%%%%%%%%%%%%%%%%%%%%%%%%%%
%%  SECTION 1: INTRODUCTION
%%%%%%%%%%%%%%%%%%%%%%%%%%%%%%%%%%%%%%%%%%%%%%%%%%%%%%%%%%%%%%%%%%%%%%%%



\section{Introduction}

\label{sec:intro}

\flag{Yike}{Plan the sections of your paper, and write very brief
notes on what you want to say in each of them. You will find it helpful
to divide them into sub-sections, and even paragraphs. You may also
want to put in placeholder figures and tables; if you do, make sure
to write the captions of them. The itemized conclusions should be
partial answers to a corresponding list of questions in the introduction.
You can write these now, as incomplete statements: when you know what
your findings are, you can complete them.}

% WL as cosmological and astrophysical probe. 


% Information is scarce. 
% Expensive, high precision future. Motivation.


% This paper: toy problem, exploring one aspect of analysis.


% History of shear estimation, and treatment of Pr(e).


Weak gravitational lensing has become the most potential tool to explore
the dark universe. One application of Weak gravitational lensing is
that it can be used to measure the mass profile of single galaxy,galaxy
groups and clusters or large scales. Cosmic shear is the mild distortion
of distant galaxy images due to the bending of light by intervening
matter.It provides one of the most promising methods for constraining
the nature of dark energy (Albrecht et al. 2006; Peacock\&Schneider
2006). While in galaxy--galaxy weaklensing, tangential shear distortions
of background galaxies around foreground galaxies, allows direct measurement
of the galaxy-DM correlation

The distortion in Weak lensing is typically ~ 0.01,or ~0.001 for
galaxy-galaxy lensing. It's usually more than 10 times smaller than
the intrinsic shape noise, it's also smaller than the distortion caused
be PSF effect which is usually around ~0.05 level. For shear measurement
the main source of uncertainty comes from the intrinsic shape noise
.Since the unlensed galaxy shape can not be observed directly the information
we have about the intrinsic ellipticity is scarce,other than  the assumption that 
the orientational angle should be isotropic. With
this assumption,an estimator can be made by simply averaging galaxies' complex
ellipticity.However by doing this we will not be able to utilize the
information about how the ellipticity are distributed. Simple average
point estimator can't use the other reasonable assumption about intrinsic
shape distribution --galaxies in different sky patches should have
the same shape distibution--which is the natural result of the homogeneous Universe.

A typical procedure of weak lensing analysis  includes ellipticity
measurement,shear inference and mass profile reconstruction. The measurement
of galaxy shapes is the first step of weak gravitational lensing analysis,since
it's the direct observable that relates with tangential shear. However
in this paper we only concern about how to infer shear from galaxy
ellipticity in a hierarchical way. Therefore in our simulation we merely
set up a simple toy model in which we have the observed ellipticity as
data,and the only measurement error we consider is a Gaussian noise
added on the lensed galaxy ellipticity.We also suppose there is a uniform shear
acting on one all the galaxies in sky patch.

Our approach is to use an exceedingly flexible model for intrinsic
shape distribution, and assumes all the galaxies in different patches
abeys such a distribution. Then we can calculate the joint likelihood
of shpape distribution parameters and shears in different sky patches.With
a classical Gibbs MCMC sampler we are able to sample their posterior
distribution.After marginalizing the shape parameters, we can obtain
the posterior distribution of shear,which we are interested in. This
approach allows us use the information maximumly without introducing
in any wrong assumption on intrinsic shape distribution.

In galaxy--galaxy lensing ,since the distortion is so weak,what people
usually do is stack many lensing signals together and get the mean
shear map,and then infers the mean mass profile of galaxy or galaxy
cluster. However with Hierarchical Bayesian method we can avoid this brute averaging
, instead we can get the posterior distribution of mass profile for
each lensing system - even though it may have very large uncertainty- we
can calculate their statistical property like galaxy-mass correlation
function via this posterior distribution.

This paper is orgnized as following: In section 2 we introduce the
weak lensing formulas and the likelihood function .Then we describe
two flexible models for the intrinsic shape distribution. In section
3 we show the result of the MCMC sampling ,and check the bias and
variance of sample mean and compare with that of the simple average
method.In Section 4 we discussed the systematical bias ,mean square error
and effects of prefactor $\xi$ and galaxy shape measurement error.

%%%%%%%%%%%%%%%%%%%%%%%%%%%%%%%%%%%%%%%%%%%%%%%%%%%%%%%%%%%%%%%%%%%%%%%%
%%  SECTION N: XXX
%%%%%%%%%%%%%%%%%%%%%%%%%%%%%%%%%%%%%%%%%%%%%%%%%%%%%%%%%%%%%%%%%%%%%%%%



\section{method}

\label{sec:XXX}


\subsection{Likelihood Functions}

We assume the lensed galaxy ellipticity $\epsilon_{\ell}$ is related
to the intrinsic galaxy ellipticity $\epsilon_{0}$ in the weak lensing
regime via 
\begin{equation}
\epsilon_{\ell}=\frac{\epsilon_{0}+g}{1+g^{*}\epsilon_{0}}
\end{equation}


and it's inverse formula 
\begin{equation}
\epsilon_{0}=\frac{\epsilon_{\ell}-g}{1-g^{*}\epsilon_{\ell}}
\end{equation}
from Schramm \& Kayser (1995), Seitz \& Schneider (1997), where $\epsilon$
is represented as a complex variable and g, $g^{*}$ are the reduced
shear and its complex conjugate, respectively. In weak lensing region we have 
\begin{equation}
g=\frac{\gamma}{1-k}\approx\gamma
\end{equation}
$\gamma$ is the shear. This is the assumption we use through out
the paper.$\epsilon$ is defined in terms of the major and minor axes and orientation a, b, $\phi$,
respectively, as
\begin{equation}
\epsilon=\frac{a-b}{a+b}e^{2i\phi}
\end{equation}
Due to the isotropicity of Universe, for the unlensed ellipticity $\epsilon_{0}$ 
we expect $\phi$ has a uniform distribution
from 0 to $\pi$.Then we can expect 
\begin{equation}
<\epsilon_{\ell}>=g
\end{equation}
Therefore $<\epsilon_{\ell}>$ can be adopted as an estimator of g.

However in our hierarchical Bayesian approach we are able to obtain
the posterior distribution of g, based on the assumption $\phi$ has
a uniform distribution and galaxies in different sky patches have
the same intrinsic shape distribution.

Suppose we know the ellipticity distribution of unlensed galaxies
based on some parasmeters $\alpha$, then we can write the likelihood
of reduced shear and $\alpha$:

\begin{equation}
p(\epsilon_{\ell}|\vec{g},\alpha)=p(\epsilon_{0}(\vec{g})|\alpha)\,|\frac{\partial\epsilon_{0}}{\partial\epsilon_{\ell}}|
\end{equation}



Note that both $\epsilon_{0}$ and $\epsilon_{\ell}$are complex numbers.So
$|\frac{\partial\epsilon_{0}}{\partial\epsilon_{\ell}}|$ is the Jacobian
determinant term,which can be calculated numerically from formula
(5).

If there is measurement error on $\epsilon_{ob}$,we will have 
\begin{equation}
p(\epsilon_{ob}|\vec{g},\alpha)=\int f(\epsilon_{ob}|\epsilon_{\ell})\, p(\epsilon_{\ell}|\vec{g},\alpha)\,d\epsilon_{\ell}
\end{equation}
Here the prior distribtuion we use for $\vec{g}$ is a 2D flat distribution
in the region $|\vec{g}|<1$.Therefore $p(\vec{g)}=1$ if $|\vec{g}|<1$
and $p(\vec{g})=0$ if $|\vec{g}|>1$.

The prior of $\alpha$ some how depends on the model we choose for
intrinsic ellipticity distribution. When we use the step function
model for P($\epsilon_{0}$), which we'll describe latter.We find that
we need to use a prior on $\alpha$. 
\begin{equation}
p(\vec{\alpha}|\xi)=exp(-\xi\sum(\alpha_{i}-\alpha_{i-1})^{2})
\end{equation}


Given the likehood functions and priors we can also calculated the
marginalized likelihood function. 
\begin{equation}
p(\epsilon_{\ell}|\vec{g})=\int p(\epsilon_{\ell}|\vec{g},\vec{\alpha})\,p(\vec{\alpha}|\vec{g})\,d\vec{\alpha}
\end{equation}


Since $p(\alpha)$ does not explicitly depends on $\vec{g}$ .$p(\alpha|\vec{g})$is
just the pror distribution $p(\vec{\alpha}|\xi)$.Therefore, we have
marginalized likehood 
\begin{equation}
p(\epsilon_{\ell}|\vec{g})=\int p(\epsilon_{\ell}|\vec{g},\vec{\alpha})\, p(\vec{\alpha}|\xi)\,d\vec{\alpha}
\end{equation}
Given that marginalized likelihood function, in principle we could
optimize it and gives out a max-likelihood estimator for $\vec{g}$
in different patches simultaneously.

One thing to note is that the likelihood of g in patch i only depends
on the shape parameters but not dependent to g in other patch.Therefore,the
marginalized likelihood of shape parameters has an equation like:

\begin{equation}
\ell(\alpha|data)=\prod_{i}\ell_{i}(\alpha|data_{i})
\end{equation}


$\ell_{i}(\alpha|data_{i})$ is the marginalized likelihood function
calculated with data in the ith patch.From this formula we can see
how better estimation of $\alpha$ will be obtained if we infer more
sky patches simultaneously. With better knowledge on $\alpha$ we
can in turn have better estimates on reduced shear g.


\subsection{Model for Intrinsic Shape distribution

The correctness of posterior distribution of g depends on the intrinsic
ellipticity distribution P($\epsilon_{0}|\alpha$). Due to the limitation
of human's knowledge we are not able to give an analytical form of
$P(\epsilon_{0}|\alpha)$.Therefore we proposed an exceedingly flexible
model for $P(\epsilon_{0}|\alpha)$, which does not requires further
assumption on $P(\epsilon_{0}|\alpha)$.

We divide the 2-D space r$\in$(0,1) into N bins .The probability
of each bin is $p_{i}=e^{\alpha_{i}}$. $\overrightarrow{\alpha}$
is a N-D parameter vector to be infered, simultaneously with g.

From practice we find that a prior smooth regularzation on $\alpha$
is needed to obtain a good estimation on shear g.

The prior distribution we used on $\alpha$ is : 
\begin{equation}
p(\vec{\alpha}|\xi)=exp(-\xi\sum(\alpha_{i}-\alpha_{i-1})^{2})
\end{equation}


Here $\xi$ is a prefactor need to be tuned .

Meanwhile,in order to show the correctness of our method, we also
set up a simulation in which we know the true $P(|\epsilon_{0}||\alpha)$
is a Beta distribution, and thus $\alpha$ is the 2-D parameter (p,q)
of the Beta distribution.We will show the posterior samples using
both of these two methods in next section. Note that for either model
we assume P($\epsilon_{0}$) is isotropic.


\subsection{Gibbs Sampler}

While making the posterior samples $p(\vec{g},\alpha|Data)$ we choose to
use the classical Gibbs sampler.

In Gibbs sampling ,samples of variable $x_{j}$ is made from the conditional
probability distribution:$p(x_{j}^{(i)}|x_{1}^{(i)},....,x_{j-1}^{(i)},x_{j+1}^{(i-1)}...,x_{n}^{(i-1)})$

The conditional probability distribution of one variable given all the other variables are proportional to their joint probability distribution

\begin{equation}
p(x_{j}|x_{1},....,x_{j-1},x_{j+1}...,x_{n})\propto p(x_{1},.....,x_{n})
\end{equation}


One thing to note is that the conditional probability of g in ith
patch only explicitly depends on $\alpha$, but not g in other patches. Therefore
with $\alpha$ fixed ,the posterior distribution of reduced shear
in ith patch are independent from each other. 
\begin{equation}
p(\vec{g}_{1}....\vec{g}_{n}|\alpha,Data)=p(\vec{g}_{1}|\alpha,Data_{1})....p(\vec{g}_{n}|\alpha,Data_{n})
\end{equation}
Therefore sampling reduced shear g from conditional distribution can
be reduced to 2D sampling problem,which can be done very quickly.

The total time needed to sample $g$ and $\alpha$ of 64 sky patches
with 8 galaxies in each patch will only be approximitly 25s on a 2
GHz CPU.


\subsection{Simulations}

The procedure of our simulation is that ,first we generated N{*}NP
unlensed galaxies' elliprticity from the ``true'' $P(\epsilon_{0})$,and
asign these galaxies into NP patches equally.So each patch has N galaxies.Second
we generate NP true reduced shears ,and asign them to each patch,and
then shear the unlensed galaxies using formula (4) .

Finally we can put on some noise on $\epsilon_{\ell}$ to get the
observed ellipticity $\epsilon_{ob}$. For simplicity we've only considered
the situation when P($\epsilon_{ob}|\epsilon_{\ell})$ is a 2D Gaussian.

Use Gibbs sampler and $P(\epsilon_{0}|\alpha)$ mentioned previously,we
are to sample the posterior distribution of g and $\alpha$ simultaneously
and effectively.

%%%%%%%%%%%%%%%%%%%%%%%%%%%%%%%%%%%%%%%%%%%%%%%%%%%%%%%%%%%%%%%%%%%%%%%%
%%  SECTION N: XXX
%%%%%%%%%%%%%%%%%%%%%%%%%%%%%%%%%%%%%%%%%%%%%%%%%%%%%%%%%%%%%%%%%%%%%%%%



\section{Result}

\label{sec:XXX}


\subsection{Posterior samples}

We show the posterior samples of both g and $\alpha$ ,for both beta
function and step function model. 

The probability distribution can be visualised by plotting points at each sample
position. An example in the x, y plane is shown for the posterior distribution of
$\vec{g}$ of 64 sky patches in Fig. .8 galaxy shape profiles in each patch is used as the data.
The number density of samples is proportional to the probability density.The figure shows that the two parameters of $\vec{g}$
do not have strong correlation,since the shape of smaples is close to a circle.
We also shows the histogram of g1 in a single sky patch. Not surprisingly it is close to a Gaussian distribution.
If the probability distribution of parameters are perfect Gaussian,their most probable estimates and uncertainty 
is just the mean and standard deviation of the Gaussian distribution.However,in pratical,we find that the distribution 
of reduced shear $\vec{g}$ will be getting closer to Gaussian,with larger number of galaxies used.When we have more than
8 galaxies in each sky patch,we found it's eligible to make such an approximation that to use sample mean and standard 
deviation as most probable estimates and uncertainty. We will justify this approximation with a bias test in the following
sub section.


\subsection{Bias Test}
In this section,we measured the levels of possible systemmatical
bias of this Hierarchical inference method.In principle given the correct model and prior ,
the hierchical Bayesian inference method would have no bias, if propagate the whole likelihood function.
However,in practical, the bias may be introduced by the model we use for the intrinsic shape distribution,
and the prior regularizer of the prefactor $\xi$ as well as the Gaussian approximation we did for the likelihood
function of $\vec{g}$. Therefore a bias test is necessary to identify how much bias there is. 
Following Heymans et al. (2006), we quantify the systematic bias in
in terms of the multiplicative error m and additive error c on the
true shear $g^{true}$ 
\begin{equation}
\hat{g_{i}}-g_{i}^{true}=m_{i}g_{i}^{true}+c_{i}
\end{equation}

To isolate out the source of the bias in this section,we assume we have perfect measurement on galaxy shape.
We will discuss more on the influence of measurement error section 4.4 .

We calculated the value of $\hat{g}-g^{true}$ of 10,000 data. Each data set has 64 patches and 8$\times$64 galaxies,and
the true P($\epsilon$) is Beta(2.8,2.8).The step function model is used with the prior prefator $\xi=3$.
The input true values of $\vec{g}_{1}$ and $\vec{g}_{2}$ of the 64$\times$10,000 sky patches are generated randomly from a uniform distribution from -0.1 to 0.1 . 

As described in the previous subsection we use the sample mean as the most probable $g_{i}$ and 
the standard deviation as the uncertainty. A standard weighted minimum $\chi^{2}$ linear fit is performed to get the
most-likely value and uncertainty of m and c.
 
Due to the limitation of galaxy number we are not able to get an exact value for m and c.However in our result 
both the most-like and uncertainty of m are around $10^{-3}$ and for c they are around $10^{-4}$ . Due to the
relatively large uncertainty, it is only an rough estimate of bias level.

\subsection{Accuracy Test}

We compare the mean square error of our estimator with average and
the estimator that could be constructed if the true unlensed ellipticity
distribution were known. We show how the mean square error varies
with number of galaxies ,number of patches and so on.


\subsection{Measurement Error}
In this section we  discuss how the shape measurement error will
affect the variance and bias of this Hierarchical Bayesian method.
In principle,if the likelihood of true lensed ellipticity $p(\epsilon_{ob}|\epsilon_{\ell})$
is known the likelihood $p(\epsilon_{ob}|g,\alpha)$ can be calculated using the convolution as equation ().
However,due to the limitation of computational time we avoid to do this convolution but use $\epsilon_{ob}$
as an estimator of $\epsilon_{\ell}$.
The likelihood of true lensed ellipticity $p(\epsilon_{ob}|\epsilon_{\ell})$ we use in the simulation is a 2D
Gaussian with mean equals $\epsilon_{\ell}$ and $\sigma$ equals 0.05 .
The result of bias test and mean square error test is shown in...




\subsection{prefactor $\xi$}

One limitation of the step function model is that it needs to be \textit{regularized};
this is achieved by assigning a prior PDF for the step heights $\vec{\alpha}$.
We use the form $Prior(\vec{\alpha}|\xi)=exp(-\xi\sum(\alpha_{i}-\alpha_{i-1})^{2})$
when inferring the P($\epsilon$) and $g$. The choice of this functional
form corresponds to the assumption that the true P($\epsilon$) is
smooth, which seems fairly natural. This function has a hyperparameter,
$\xi$, that needs to be set somehow. If $\xi$ is too small, the
inferred P($\epsilon$) will be too noisy, and the accuracy of the
$g$ inference will decrease. Likewise, choosing an $\xi$ that is
too large, the inferred P($\epsilon$) will be too smooth, and the
data will be under-fitted. In principle one could infer $\xi$ form
the data at hand, but in practice we find that we further more need some
prior regularizer on $\xi$ to obtain correct inference.Therefore in this paper 
we just tuned $\xi$ empirically .From various trials, We find
that $\xi$ should be proportional to the mean data number density and can
be approximated by the following empirical formula: 
\begin{equation}
\xi=N<\rho(|\epsilon|)>/256*\frac{Nbin}{10}
\end{equation}
Here $<\rho|\epsilon|>=\int\rho(|\epsilon|)^{2}d(|\epsilon|)$ corresponds
to the average of data number density,and Nbin is the number of bins used 
in step function model. 

%%%%%%%%%%%%%%%%%%%%%%%%%%%%%%%%%%%%%%%%%%%%%%%%%%%%%%%%%%%%%%%%%%%%%%%%
%%  SECTION N: XXX
%%%%%%%%%%%%%%%%%%%%%%%%%%%%%%%%%%%%%%%%%%%%%%%%%%%%%%%%%%%%%%%%%%%%%%%%



\section{Discussion}

\label{sec:XXX} In weak lensing survey,one way to obtain high S/N
ratio is to stack large number of lensing signal together,and calculate
the mean shear map of all these lensing systems.However simply to
stack all the lensing signals would not be the optimal way because
of the amount of information we can obtain from each lensing system
can be very different. The Bayesian inference approach we proposed
allows a natural way to optimally include all the data.In stead of
stacking what we need to do is to propogate the likelihood of the
shear map of each lensing system. The statistical property of all
the lensing systems like the galaxy-mass ,shear-shar correlation function
,in principle can be calculated from these likelihood functions.



%%%%%%%%%%%%%%%%%%%%%%%%%%%%%%%%%%%%%%%%%%%%%%%%%%%%%%%%%%%%%%%%%%%%%%%%
%%  SECTION X: CONCLUSIONS
%%%%%%%%%%%%%%%%%%%%%%%%%%%%%%%%%%%%%%%%%%%%%%%%%%%%%%%%%%%%%%%%%%%%%%%%



\section{Conclusions}

\label{sec:conclusions}

In summary, we have presented a conceptually hierachical inference
approach for weak lensing shear estimation, with carefully calculated 
uncertainty estimates.This method is intrinsically cope with high noise
levels on the input data and naturally allows bad fit to be identified 
from the residuals map. We also have proposed a flexible non-parametric model 
for intrinsic shape distribution allows us to obtain the correct posterior 
distributions, while avoiding making any unsupported assumptions.The Gibbs
sampler allows us to sample the posterior distribution of shear in large number of
sky patches simultaneously and efficiently.

Several  model fitting galaxy shape measurent method have been proposed in recent years.
Unlike the average method which merely request a point estimator of galaxy ellipcity,our 
hierachical bayesian approach utilizes the whole likelihood of the galaxy ellipticity.
\begin{itemize}
\item We ...
\item The ...
\item The ...
\end{itemize}
%%%%%%%%%%%%%%%%%%%%%%%%%%%%%%%%%%%%%%%%%%%%%%%%%%%%%%%%%%%%%%%%%%%%%%%%
%%  ACKNOWLEDGMENTS
%%%%%%%%%%%%%%%%%%%%%%%%%%%%%%%%%%%%%%%%%%%%%%%%%%%%%%%%%%%%%%%%%%%%%%%%



\section*{Acknowledgments}

We thank XXX for useful discussions and suggestions.

YT acknowledges ....
%
DWH acknowledges support from ...
% 
PJM was given support by the Royal 
Society in the form of a research fellowship.
%
% Code used? Links to repositories?


%%%%%%%%%%%%%%%%%%%%%%%%%%%%%%%%%%%%%%%%%%%%%%%%%%%%%%%%%%%%%%%%%%%%%%%%
%%  APPENDICES
% %%%%%%%%%%%%%%%%%%%%%%%%%%%%%%%%%%%%%%%%%%%%%%%%%%%%%%%%%%%%%%%%%%%%%%
% 
% \appendix
% 
% \section{}
% \label{sec:appendix}
% 
% 
%%%%%%%%%%%%%%%%%%%%%%%%%%%%%%%%%%%%%%%%%%%%%%%%%%%%%%%%%%%%%%%%%%%%%%%%
%%  REFERENCES
%%%%%%%%%%%%%%%%%%%%%%%%%%%%%%%%%%%%%%%%%%%%%%%%%%%%%%%%%%%%%%%%%%%%%%%%


% MNRAS does not use bibtex, input .bbl file instead. 
% Generate this in the makefile using bubble script in scriptutils:


%   bubble -f PS1QLS-survey.tex references.bib 


 \bibliographystyle{apj}
\bibliography{references}
 %\input{PS1QLS-survey.bbl}


%%%%%%%%%%%%%%%%%%%%%%%%%%%%%%%%%%%%%%%%%%%%%%%%%%%%%%%%%%%%%%%%%%%%%%%%


\label{lastpage} \bsp

\end{document}

%%%%%%%%%%%%%%%%%%%%%%%%%%%%%%%%%%%%%%%%%%%%%%%%%%%%%%%%%%%%%%%%%%%%%%%%

\end{document}
