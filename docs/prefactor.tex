%% LyX 2.0.4 created this file.  For more info, see http://www.lyx.org/.
%% Do not edit unless you really know what you are doing.
\documentclass[english]{article}
\usepackage[T1]{fontenc}
\usepackage[utf8]{luainputenc}
\usepackage{esint}
\usepackage{babel}
\begin{document}
One limitation of the step function model is that it needs to be {\it regularized}; 
this is achieved by assigning a prior PDF for the step heights $\alpha$.
We use the form
$Prior(\alpha)=exp(-\xi\sum(\alpha_{i}-\alpha_{i-1})^{2})$ 
% alpha should be a vaector here, and you should write the prior as P(alpha|\xi) to show the dependence on the 
% regularisation constant.
when inferring the P($\epsilon$) and $g$.
The choice of this functional form corresponds to the assumption
that the true P($\epsilon$) is smooth, which seems fairly natural. This function has a hyperparameter, 
$\xi$, that needs to be set somehow. If $\xi$ is too small, the inferred P($\epsilon$)
will be too noisy, and the accuracy of the $g$ inference will
decrease. Likewise, choosing an $\xi$ that is too large, the inferred P($\epsilon$) will be too smooth, and the 
data will be under-fitted. In principle one could infer $\xi$ form the data at hand, but in practice we find...
% What happened when you tried to infer \xi? Why does this not work?
We find that
$\xi$ is proportional to the mean data density and can be approximated
by the following empirical formula:
\begin{equation}
\xi=<\rho>/256
\end{equation}
% What is the "data density" and why does it appear squared here?
Here $<\rho>=\int Nf^{2}(x)dx$ is the mean data density. With the
help of this empirical formula, we can tune this prefactor $\xi$ 
by doing some simulation which has similar data distribution with the real data.
% This needs better wording: justifying your choices is very important!
\end{document}
